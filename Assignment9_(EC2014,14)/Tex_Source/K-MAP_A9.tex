\documentclass{article}
\usepackage[utf8]{inputenc}
\usepackage{graphicx}
\usepackage{karnaugh-map}

\title{K-MAP\\Assignment9}
\author{Abhay Suresh}
\date{4th January 2021}

\begin{document}

\maketitle
\section {EC-2014,14}
\begin{center}
\begin{karnaugh-map}[4][2][1][][]
    \maxterms{0,1,2,4,6,7}
    \minterms{3,5}
    \implicant{3}{3}
    \implicant{5}{5}
    % note: position for start of \draw is (0, Y) where Y is
    % the Y size(number of cells high) in this case Y=2
    \draw[color=black, ultra thin] (0, 2) --
    node [pos=0.7, above right, anchor=south west] {$YZ$} % YOU CAN CHANGE NAME OF VAR HERE, THE $X$ IS USED FOR ITALICS
    node [pos=0.7, below left, anchor=north east] {$X$} % SAME FOR THIS
    ++(135:1);
    \end{karnaugh-map}
    \end{center}

\end{document}
